
%%--------------------------------%%
\begin{frame}[ctb!]
    \frametitle{Blue Ribbon Commission Charter}
    \footnotesize{
      The goals of the blue ribbon commission were \cite{matthews_blue_2010}:
    \begin{itemize}
      \item Evaluation of existing fuel cycle technologies and R&D programs. 
      \item Options for safe storage of used nuclear fuel while final 
        disposition pathways are selected and deployed.
      \item Options for permanent disposal of used fuel and/or high-level nuclear waste, 
      \item Options to make legal and commercial arrangements for the management 
        of used nuclear fuel and nuclear waste in a manner that takes the 
        current and potential full fuel cycles into account.
      \item Options for decision-making processes for management and disposal 
        that are flexible, adaptive, and responsive. 
      \item Options to ensure that decisions on management of used nuclear fuel 
        and nuclear waste are open and transparent, with broad participation.
      \item The possible need for additional legislation or amendments to 
        existing laws, including the Nuclear Waste Policy Act of 1982, as 
        amended. 
    \end{itemize}
  }
  \end{frame}

%%--------------------------------%%
\begin{frame}[ctb!]
    \frametitle{Recommendations from the BRC}
    \footnotesize{
    The BRC came out with a list of recommendations 
    \cite{ayers_report_2012} :
    \begin{itemize}
      \item Consent‐based siting approach 
      \item An authoritative organization dedicated solely to implementing the
waste management program.
      \item Access to the waste fund.
      \item Prompt efforts : one or more geologic disposal facilities.
      \item Prompt efforts : one or more consolidated storage facilities.
      \item Prompt efforts : to prepare for SNF and HLW transportation.
      \item Supportfor continued U.S. innovation in nuclear energy technology and for workforce development; 
      \item U.S. leadership in international efforts to address safety, waste management, non‐proliferation, and security concerns.
    \end{itemize}
 }
  \end{frame}

%%--------------------------------%%
\begin{frame}[ctb!]
    \frametitle{DOE Response to the BRC}

    DOE created four working groups. 
    \begin{itemize}
      \item Governance Framework and Funding
      \item System Design \& Architecture
      \item ConsentBased Siting
      \item Transportation Routing, Safety and Security
    \end{itemize}
  \end{frame}

%%--------------------------------%%
\begin{frame}[ctb!]
    \frametitle{DOE Response to the BRC}
    \footnotesize{
    In a brief statement \cite{doe_strategy_2013}, the DOE agreed in general 
    with the BRC and said that DOE would pursue a timeline :

    \begin{itemize}
      \item Sites, designs and licenses, constructs and begins operations of a 
        \textbf{pilot interim storage facility} by \textbf{2021} with an initial 
        focus on accepting used nuclear fuel from shut-down reactor sites.
      \item Advances toward the siting and licensing of a \textbf{larger interim 
        storage facility} to be available by \textbf{2025}.
      \item Makes \textbf{demonstrable progress} on the siting and characterization of 
        repository sites to facilitate the availability of a geologic repository 
        by \textbf{2048}.
    \end{itemize}
  }
  \end{frame}

%%--------------------------------%%
\begin{frame}[ctb!]
    \frametitle{Congressional Response to the BRC}
    Senator Jeff Bingaman, before leaving office, introduced a bill called the 
    \textbf{Nuclear Waste Administration Act of 2011}. It suggested an independent waste fund management body of the administrative type.  The future of this bill is unknown.  
  \end{frame}

