
%%----------------------------------------%%
\begin{frame}[ctb!]
  \frametitle{Heat Contributors In PWR SNF}
\footnotesize{
  \begin{figure}[htbp!]
  \begin{center}
    \includegraphics[width=0.7\textwidth]{wigeland_heat.eps}
  \end{center}
  \caption{Heat contributors in a canonical PWR 
    fuel\cite{wigeland_relationship_2010}.}
  \label{fig:<++>}
\end{figure}

}
\end{frame}

%%----------------------------------------%%
\begin{frame}[ctb!]
  \frametitle{Heat Contributors in PWR SNF}
\footnotesize{
  \input{carter_pwr_heat_fig}
}
\end{frame}
%%----------------------------------------%%
\begin{frame}[ctb!]
  \frametitle{Heat Contributors in LWR Recycled MOX}
\footnotesize{
  \input{carter_lwr_mox_heat_fig}
}
\end{frame}
%%----------------------------------------%%
\begin{frame}[ctb!]
  \frametitle{Heat Contributors After NUEX Recycling}
\footnotesize{
  \input{carter_nuex_heat_fig}
}
\end{frame}

%%----------------------------------------%%
\begin{frame}[ctb!]
  \frametitle{Heat Contributors After COEX Recycling}
\footnotesize{
  \input{carter_coex_heat_fig}
}
\end{frame}
%%----------------------------------------%%
\begin{frame}[ctb!]
  \frametitle{Summary: Heat Contributing Isotopes in Various Fuel Cycles}
\footnotesize{
Dominant thermal contributors vary among fuel cycles. 
\begin{itemize}
   \item Recycling schemes are likely to reduce transuranics and actinides.
   \item Fission products such as Cs and Sr are powerful heat contributors in the first 1000 years, when capacity limiting peak heat is likely to occur in most geologies.
   \item Transuranics, Pu, Np, Am, and Cm are dominant long term heat contributors. Some extraction processes are more successful at removing those from the waste stream. 
\end{itemize}
}
\end{frame}
