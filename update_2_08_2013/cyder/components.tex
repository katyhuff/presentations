
\begin{frame}[ctb!]
  \frametitle{Demonstration Case : Code Development}
  \footnotesize{
  The demonstration case is an empty software architecture in which to implement 
  the physical models. This demonstration has built and tested
  \begin{itemize}
    \item component module loading of models and data
    \item information passing between modules
    \item and database writing.
  \end{itemize}
  }
  \begin{figure}[htbp!]
    \begin{center}
      \includegraphics[height=.5\textheight]{cyder/images/componentLoading.eps}
    \end{center}
  \end{figure}
\end{frame}

\begin{frame}[ctb!]
  \frametitle{System Level Abstraction}
  \begin{figure}[h!]
      \includegraphics[width=\textwidth]{cyder/images/abstractionSystem.eps}
    \caption{System level abstraction seeks to determine the systems level 
    response to the change in models of subcomponents.}
  \end{figure}
\end{frame}

\begin{frame}
  \frametitle{Nested Components}
  Each component represents a 
  \begin{itemize}
    \item Waste Form
    \item Waste Package
    \item Buffer
    \item or Far Field (geologic medium).
  \end{itemize}
\end{frame}


\begin{frame}
  \frametitle{Nested Components}
  Each Component has : 
  \begin{itemize}
    \item a Geometry to describe its dimensions and location
    \item a NuclideModel for contaminant transport 
    \item a ThermalModel for heat transport
    \item a Parent Component at its external barrier
    \item one or more Daughter Components at its internal barrier
  \end{itemize}

  Components have other data members such as a Type (WF, WP, BUFFER, FF), a 
  material data table, a start date, etc. 
\end{frame}

\begin{frame}
  The NuclideModel in a Component can be interchangeably represented by any of 
  the four nuclide transport models. 
    \begin{itemize}
      \item Degradation Rate Based Failure Model
      \item Mixed Cell with Degradation, Sorption, Solubility Limitation
      \item Lumped Parameter Model
      \item 1D Advection Dispersion Solution
    \end{itemize}
\end{frame}
