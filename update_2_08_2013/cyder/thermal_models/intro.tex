% Provide a summary of the work conducted:
%      Describe the technical problem clearly
%      support it with a method

\begin{frame}[ctb!]
\frametitle{Thermal Modeling in Cyder}
Two types of thermal modeling occur in Cyder. 
\begin{itemize}
\item The first is \textbf{capacity estimation} for waste stream acceptance.
\begin{itemize}
\item It employs a \gls{STC} algorithm \cite{radel_effect_2007, radel_repository_2007} and
\item relies on a supporting \textbf{response database} combining detailed 
spent nuclear fuel composition data \cite{carter_fuel_2011} with a detailed 
thermal repository performance analysis tool from \gls{LLNL} and the \gls{UFD} 
campaign \cite{greenberg_application_2012}.  
\end{itemize}
\item The next is \textbf{heat evolution} which (optionally) determines heat evolution in 
the modules over repository lifetime.
\end{itemize}
\end{frame}


\begin{frame}[ctb!]
\frametitle{Thermal Modeling in Cyder}
\begin{itemize}
\item The \textbf{capacity estimation} method is capable of rapid estimation of 
temperature increase near emplacement tunnels as a function of waste 
composition, limiting radius, $r_{lim}$, waste package spacing, $S$, near field 
thermal conductivity, $K_{th}$, and near field thermal diffusivity, 
$\alpha_{th}$.

\item The \textbf{heat evolution} in the modules includes a lumped parameter 
model in addition to the \gls{STC} method.
\end{itemize}

\end{frame}


\begin{frame}[ctb!]
\frametitle{Thermal Modeling in Cyder}
\footnotesize{

Waste conditioning is the process of packing a waste stream into an appropriate 
waste form. A waste stream is accepted into the repository, if it meets thermal capacity requirements. 
It is according to user specified waste form  
pairings that Cyder loads discrete waste forms with discrete waste 
stream contaminant vectors as depicted in Figure \ref{fig:ws_conditioning}.

\begin{figure}[htbp!]
\begin{center}
\def\svgwidth{.5\textwidth}
\input{./cyder/thermal_models/ws_conditioning.eps_tex}
\end{center}
\caption{Waste streams are conditioned into the appropriate waste form 
according to user-specified pairings.}
\label{fig:ws_conditioning}
\end{figure}

  }
\end{frame}

\begin{frame}[ctb!]
\frametitle{Thermal Modeling in Cyder}
\footnotesize{


Waste packaging is the process of placing one or many waste forms into a 
(typically metallic) containment package. Once the waste stream has been 
conditioned into a waste form, that waste form Component is loaded into a waste 
package Component, also according to allowed pairs dictated by the user, as 
depicted in Figure \ref{fig:wf_packaging}.

\begin{figure}[htbp!]
\begin{center}
\def\svgwidth{.5\textwidth}
\input{./cyder/thermal_models/wf_packaging.eps_tex}
\end{center}
\caption{Waste forms are loaded into the appropriate waste package 
according to user-specified pairings.}
\label{fig:wf_packaging}
\end{figure}
  }
\end{frame}

\begin{frame}[ctb!]
\frametitle{Thermal Modeling in Cyder}
\footnotesize{

Finally, the waste package is emplaced in a buffer component, which 
contains many other waste packages, spaced evenly in a grid. The grid is 
defined by the user input and depends on repository depth, $\Delta z$, waste 
package spacing, $\Delta x$, and tunnel spacing, $\Delta y$ as in Figure 
\ref{fig:repo_layout}.

\begin{figure}[htbp!]
\begin{center}
\def\svgwidth{.5\textwidth}
\input{./cyder/thermal_models/repo_layout.eps_tex}
\end{center}
\caption{The repository layout has a depth and a uniform package spacing.}
\label{fig:repo_layout}
\end{figure}

  }
\end{frame}
