
\begin{frame}[ctb!]
\frametitle{<++>}
\footnotesize{
<++>
}
\end{frame}

\begin{frame}[ctb!]
\frametitle{<++>}
\footnotesize{
<++>
}
\end{frame}


\begin{frame}[ctb!]
\frametitle{<++>}
\footnotesize{
<++>
}
\end{frame}


The lumped heat capacitance model makes an analogy to electrical circuit by 
reducing a thermal system into discrete lumps for an approximate solution of 
transient heat transfer. Such an approximation is 
appropriate when it can be assumed that the temperature gradient within each 
lump is approximately uniform. The appropriateness of this approximation can be
quantitatively expressed by comparison of the internal thermal resistance to the 
external thermal resistance. The Biot number, 

\begin{align}
  Bi = \frac{hA}{k}
  \label{biot}
\end{align}
indicates the relative speeds with which heat conducts within an object and 
across the boundary of that object. If the Biot number is low $(<0.1)$, and 
therefore conduction is faster within the object than at the boundary, the 
assumption of a uniform internal temperature is appropriate and the lumped 
parameter model may be expected to give a result within $5\%$ 
error\cite{incropera_fundamentals_2006}. This assists in choosing the size of 
distinct lumps within a conceptual model. 

The lumped capacitance model can address multiple media and multiple heat
transfer modes. The rate of heat transfer $\dot{q}\hspace{1mm}[Wm^{-2}K^{-1}s^{-1}]$ 
through a circuit is simply given as the quotient of the temperature 
difference and the sum of thermal resistances, $R_i [W\cdot K^{-1}]$,
of the multiple lumps 

\begin{align}
  \dot{q} = \frac{\Delta T}{\sum _{i=0}^{N}R_i}.
\end{align}

By representing the various modes of heat transport (i.e. conduction, 
convection, radiation, and mass transfer) with various expressions for 
resistance, the lumped capacitance model provides a solution to the transient 
problem described by the energy balance,

\begin{align}
  \left( \mbox{Energy added to body j in dt} \right) &= \left( \mbox{Heat 
  out of adjacent bodies into body j} \right)\nonumber\\
  c_j\rho_j V_j dT_j(t) &= \sum_{i=0}^{i=N}\left[\dot{q_{i,j}}\right]dt,
\end{align}

where $c_j\rho_jV_j$ is the total lumped thermal capacitance of the body.

For example, in the case of a simple convective circuit between two bodies, $i$ 
and $j$, the resistance of $j$ can be described as 
\begin{align}
  R_{conv} &= 1/hA
  \intertext{such that}
  c_j\rho_j V_j dT_j(t) &= \sum_{i=0}^{i=N}\left[ hA_j (T_i - T_j(t)) \right]dt.\\
\end{align}

A time constant appears under integration that describes the speed with which 
the body $i$ changes temperature with respect to the maximum temperature change,

\begin{align}
  \int_{T_j=T_0}^{T_j(t)} 
  \frac{dT_j(t)}{T_i-T_j}&=\frac{hA_j}{c_j\rho_jV_j}\int_0^t dt\\
  -ln\frac{T_i-T_j(t)}{T_i-T_0}&=\frac{hA_j}{c_j\rho_jV_j}t\\
  \frac{T_i-T_j(t)}{T_i-T_0}&= e^{-(hA_j/c_j\rho_jV_j)t}\\
  \intertext{such that}
  \frac{T_j(t)-T_i}{T_i-T_0}&= 1- e^{-t/\tau}\\
  \intertext{where}
  \tau &= \left( c_j\rho_jV_j/hA_j \right).
\end{align}

The time constant, $\tau$ is the time it takes for the body to change 
$(1-(1/e))\%\Delta T$ and is equal to the product of the thermal capacitance and 
thermal resistance of the body, $CR$, analogous to an electrical circuit.

\cite{el-wakil_nuclear_1981} This is the case for all resistances, $R_i$ 
representing modes of heat transfer. Thus, one can say, in general

\begin{align}
  \tau_j = c_j \rho_j V_j R_j.
\end{align}
